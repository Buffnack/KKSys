\documentclass[
a4paper,
twoside,
DIV=12,
BCOR=8mm,
headlineinclude=true,
footinclude=false,
numbers=noenddot,
headheight=40pt, 11pt]{scrartcl}

\usepackage[T1]{fontenc}
\usepackage[utf8]{inputenc}
\usepackage[ngerman]{babel}
\usepackage{amsfonts, amsmath, amsthm, amssymb}
\usepackage{graphicx}
\usepackage{multicol}
\usepackage{fancyhdr}
\usepackage{listings}
\pagestyle{fancy}
\setlength{\parindent}{0pt}

\lhead{NameNotFound}
\rhead{Datenbankmodel}

\begin{document}
	\section*{Datenbank Entwurf}
	\subsection[Tags]{Tabellen ohne Normalisierung: Tags}
	\begin{tabular} {|c| c|}
		ID & Tag  \\
	\end{tabular} 
	\subsection[Label]{Tabelle ohne Normalisierung: EventLabel}
	\begin{tabular}{|c| c| c|}
		ID & Name & OberLabel? \\
	\end{tabular}
	\subsection[Events]{Tabelle ohne Normalisierung: Events}
	Event:
	\begin{tabular}{|c| c| c |c |c |c|c|}
		ID & Start & Ende& \textbf{Label}& Name & DayCode & Repeating? 
	\end{tabular}\\ Eventlabel:
	 \begin{tabular}{|c |c|}
	 	ID & Name
	 \end{tabular}
 	\subsection[Thema]{Tablle ohne Normalisierung: Thema}
 	\begin{tabular}{|c| c| c|}
 		ID & Name & OberThema?
 	\end{tabular}
 	\subsection[Karteikarte]{Tabelle ohne Normalisierung: Karteikarten und Art} Karteiart:
 	\begin{tabular}{c |c |}
 		ID & Art
 	\end{tabular} \\ Karteikarten:
 	\begin{tabular}{|c| c| c| c| c| c|}
 		ID & \textbf{Thema} & Object & \textbf{Tag} & \textbf{EventLabel} & \textbf{KarteikartenArt} 
 	\end{tabular}
 \\ \\
 Nutzung von SQL Like wird uns helfen die einzelnen Veranstaltungen welche an mehreren Tagen laufen zu finden.
\end{document}